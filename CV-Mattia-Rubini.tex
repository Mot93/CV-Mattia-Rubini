\documentclass[%singlesided,
               doublesided,
               paper=a4,
               fontsize=10pt
              ]{Rubini-Mattia-resume}


%%%%%%%%%%%%%%%%%%%%%%%%%%%%%%%%%%%%%%%%%%%%%%%%%%%%%%%%%%%%%%%%%%%%%%%%%%%%%%%%
% set geometry
%%%%%%%%%%%%%%%%%%%%%%%%%%%%%%%%%%%%%%%%%%%%%%%%%%%%%%%%%%%%%%%%%%%%%%%%%%%%%%%%

\setlength\highlightwidth{7cm}
\setlength\headerheight{4cm}            % note that margintop gets added to this value, i.e. the header bar is 5cm
\setlength\marginleft{1cm}
\setlength\marginright{\marginleft}      % needs to be 1.5 times to be actually equal. why?
\setlength\margintop{0cm}
\setlength\marginbottom{1cm}


%%%%%%%%%%%%%%%%%%%%%%%%%%%%%%%%%%%%%%%%%%%%%%%%%%%%%%%%%%%%%%%%%%%%%%%%%%%%%%%%
% FONTS
%%%%%%%%%%%%%%%%%%%%%%%%%%%%%%%%%%%%%%%%%%%%%%%%%%%%%%%%%%%%%%%%%%%%%%%%%%%%%%%%

\RequirePackage{fontspec}
\setmainfont{Carlito}

%%%%%%%%%%%%%%%%%%%%%%%%%%%%%%%%%%%%%%%%%%%%%%%%%%%%%%%%%%%%%%%%%%%%%%%%%%%%%%%%
% COLORS
%%%%%%%%%%%%%%%%%%%%%%%%%%%%%%%%%%%%%%%%%%%%%%%%%%%%%%%%%%%%%%%%%%%%%%%%%%%%%%%%

\colorlet{highlightbarcolor}{lightgray}
\colorlet{headerbarcolor}{darkgray}

\colorlet{headerfontcolor}{white}
\colorlet{accent}{awesome-red}
\colorlet{heading}{black}
\colorlet{emphasis}{black}
\colorlet{body}{black}


%%%%%%%%%%%%%%%%%%%%%%%%%%%%%%%%%%%%%%%%%%%%%%%%%%%%%%%%%%%%%%%%%%%%%%%%%%%%%%%%
% set document
%%%%%%%%%%%%%%%%%%%%%%%%%%%%%%%%%%%%%%%%%%%%%%%%%%%%%%%%%%%%%%%%%%%%%%%%%%%%%%%%


\begin{document}

\name{Mattia Rubini - DevOps}
\tagline{Il bello dei computer è che fanno esattamente quello che gli dici di fare. 
	\\Il brutto dei computer è che fanno esattamente quello che gli dici di fare.}
\photo[round]{photo-Mattia.jpg}{\dimexpr \headerheight-\marginbottom}   % make photo exactly match the header with margintop/marginright/marginbottom as margin

\makeheader

\highlightbar{

    \section{Contatti}
    
    \email{mattia.rubini@gmail.com}
    \phone{+39 329 617 1153}
    \location{Via L. Spada 30/A, 40129 Bologna}
    \vspace{0.5em}
    \github{Mot93}{https://github.com/Mot93}
    \linkedin{Mattia Rubini}{https://www.linkedin.com/in/mattia-rubini-1b8b49112/}
    
    \section{Skills}

    %\vspace{0.5em}
    %\section{Certificates}
    %\simpleskill{Kubernetes Administrator}
    
    \vspace{0.5em}
    \skillsection{Cloud Native}
    \skill{Linux}{4}
    \skill{kubernetes}{4}
    \skill{Docker/Podman}{5}
    \skill{Azure}{3}
    \skill{AWS}{1}
    \skill{Istio}{2}
    \skill{Helm}{4}
    \skill{ArgoCD}{3}
    
    \vspace{0.5em}
    \skillsection{DevOps}
    \skill{Git}{4}
    \skill{Ansible}{5}
    \skill{Jenkins}{3}
    \skill{GitLab}{3}
    \skill{Azure DevOps}{2}
    \skill{Terraform}{4}
    
    \vspace{0.5em}
    \skillsection{Programming}
    \skill{Python}{4}
    \skill{Rust}{2}
    \skill{SQL}{5}
    \skill{JavaScript}{3}
    \skill{Dart}{2}

    \vspace{0.5em}
    \skillsection{Frameworks}
    \skill{Python Django}{4}
    \skill{Rust Rocket}{2}
    \skill{Dart Flutter}{2}

    \vspace{0.5em}
    \skillsection{Altre skills}
    \skill{Ubiquiti}{4}
    \skill{Raspberry Pi}{4}
    \skill{Data science}{2}
    \skill{Nginx}{3} 

    \vspace{0.5em}
    \skillsection{Languages}
    \skill{Italian}{5}
    \skill{English}{4}
    \skill{French}{4}
    \bigskip

}
\mainbar{
    \section{Presentazione}
    	La mia passione per l'informatica nasce il primo anno di liceo quando vengo introdotto per la prima volta alla programmazione.\
	Da allora, cerco sempre di migliorarmi e di imparare cose nuove.
	Tutti i progetti realizzati fino ad oggi sono frutto sia del mio lavoro da autodidatta sia delle mie esperienze lavorative.
    
    \section[\faGears]{Esperienze Lavorative}
    \job{Apr 2021 - Oggi}
        {AliasLab SRL}
        {DevOps Engineer}
        {Impiegato presso AliasLab come DevOps Engineer e Cloud Engineer. 
    Incaricato di progettare nuovi prodotti cloud native e la corrispettiva parte di: DevOps, Site Reliability e Security.
    }

    \job{Apr 2020 - Oggi}
        {VCube SRL}
        {System Architect}
        {Impiegato presso VCube come DevOps e System Architech.
    Assegnato a consegne dedicate alla gestione di server Windows e Linux e all'automazione di processi di deployment tramite l'uso di \href{https://www.ansible.com/}{Ansible}, \href{https://www.jenkins.io/}{Jenkins} e \href{https://www.python.org/}{Python}.
    }

    \job{Giu 2019 - Apr 2020}
        {CED. IS SRL}
        {System Integrator}
        {Impiegato presso il MAIS family office ho incominciato il percorso di System Integrator. 
	Finita la mia formazione, ho incominciato ad espandermi in altri ruoli: Network Engineer, System Administrator, Data Scientist e DevOps.
    }
    
    \section[\faMortarBoard]{Formazione}
    \job{2020 - Oggi}
        {Alma Mater Studiorum, Bologna}
        {Laureando in Informatica}
        {Corso di studi in scienze informatiche.}
    \job{2017 - 2020}
        {Alma Mater Studiorum, Bologna}
        {Laureando in Ing. Informatica}
        {Interrotto, per trasferimento ad Informatica. superati 10 esami.}
    \job{2014 - 2015}
        {Alma Mater Studiorum, Bologna}
        {Borsa di studio a Shangai}
        {Gemelaggio tra Ing. dell'Automazione di Bologna e Shangai}
    \job{2013 - 2016}
        {Alma Mater Studiorum, Bologna}
        {Laureando in Ing. dell'automazione}
        {Interrotto, per trasferimento ad Ingegneria Informatica. superati 15 esami.}
    \job{2007 - 2013}
        {Liceo Niccolò Copernico, Bologna}
        {Diploma Liceo Scientifico}
        {Indirizzo di studi MAXI (potenziamento di liceo scientifico, su materie: matematica, fisica, programmazione e scienze).
	Tale indirizzo insegna a programmare fin dai primi anni.}

\section{Achievements}
    \achievement{HomeLab\\
        Ho progettato e implementato la mia infrastruttura informatica e domotica usando prodotti \href{https://www.ui.com/}{Ubiquiti}, \href{https://www.qnap.com}{QNap} (NAS), \href{https://www.raspberrypi.org/}{Raspberry Pi} (home server e cluster) e \href{https://www.ikea.com/it/it/p/tradfri-gateway-bianco-40337806/}{Ikea}.
        Ho anche predisposto tutta una serie di laboratori dedicati alla mia formazione da autodidatta (cluster kubernetes, server Jenkins, etc...)
    }


    %\section{General Skills}
    %\smallskip % additional skip because tag outlines use up space
    %\tag{Tag 1}
    %\tag{Tag 2}
    %\tag{and}
    %\tag{another tag}
    %\tag{some more tags}
    %\tag{yet another one}
    %\tag{tags flow over}
    %\tag{to the next line}
    %\tag{if necessary}
    
    %\medskip
    %Tags must be ordered by hand with newlines to get a nice layout, especially for long tags.
    
    %\section{Wheel Chart}
    % This is taken from AltaCV
    % see https://github.com/liantze/AltaCV for details
    %\wheelchart{1.5cm}{0.5cm}{% outer and inner diameter
    %    6/8em/accent!20/Sleep,          % comma-separated list of
    %    8/8em/accent!40/Daytime job,    % fraction of 24 / line length / color / label
    %    2/8em/accent!80/Training,          % here, the color is shades of the accent color
    %    3/8em/accent!60/Recovering from fighting criminals,
    %    5/8em/accent/Being Batman
    %}
}
\makebody
\clearpage


%\pagestyle{highlightmain}
%\highlightbar{}
%\mainbar{

    %\section{Another section}
    
    %This page uses the page style \texttt{highlightmain} which shows the highlight bar (gray) and the main part (white background) but omits the header. 
    %The default page style is \texttt{headerhighlightmain} with all three elements.
    %If you don't want header, nor highlight bar, use page style \texttt{\textbackslash pagestyle\{empty\}}.
    %\medskip
    %Neither main, nor highlight bar must be filled to make this template work.
    %It is possible to use a page style with the highlight bar but leave it empty by setting an empty highlightbar \texttt{\textbackslash highlightbar\{\}}.

    %\vspace{0.5em}
    %\subsection{Subsection 1}
    %Demonstrate subsections.
    
    %\subsection{Subsection 2}
    %Subsection are also bold face but a smaller font then section. They also omit the rule.
    

%}
%\makebody


%\clearpage
\pagestyle{empty}

    \section{Altre Esperienze}
        Membro del \href{https://en.wikipedia.org/wiki/Rotaract}{Rotaract} Club Bologna Valle del Savena.
        \\\textbf{System administrator del distretto \href{https://rotaract2072.com/}{Rotaract 2072}.}
        \\Assemblo computer per uso personale, amici e familiari.
        \\ \textbf{Il mio interesse per Raspberry Pi e Arduino ha trovato applicazione sia nella supervisione degli apparati tecnologici del MAIS family office, sia per la mia abitazione.}
        \\Vacanza studio-lavoro di un mese con attività di volontariato per la ripopolazione della vegetazione in Islanda (estate 2011).
        \\Vacanza studio-lavoro di un mese con attività di volontariato per la preparazione di una struttura (situata in Inghilterra) dedicata ad aiutare ragazzi appartenenti a famiglie con problemi socio economici (estate 2010).

%\section{Publications}
%\pubforcefullwidth

%Demonstrate what an \texttt{\textbackslash pagestyle\{empty\}} page looks like.
%Also show off the macros for publications that uses small icons for authors, date, journal and links.

%Achieving a good looking spacing can be tricky. For empty pagestyles where the full width is available use \texttt{\textbackslash pubforcefullwidth} to force the publoication list %to take up all the available space.
%The (relative) lengths reserved for date, journal and links can be set with the parameters \texttt{\textbackslash pubdatelength}, \texttt{\textbackslash pubjournallength} and \texttt{\textbackslash publinklength} as in \texttt{\textbackslash setlength\{\textbackslash pubdatelength\}\{0.15 \textbackslash linewidth\}}.
%\bigskip

%\publication
%	{The turbulent gas structure in the centers of NGC~253 and the Milky Way} % Title
%	{\textbf{N. Krieger}, A. Bolatto, E. Koch, A. Leroy, E. Rosolowsky, F. Walter, A. Wei\ss, D. Eden, R. Levy, D. Meier, E. Mills, T. Moore, J. Ott, Y. Su, S. Veilleux} % Authors
%	{2020} % Year
%	{The Astrophysical Journal Vol. 899, Issue 2, id.158} % Journal
%	{\ADS{https://ui.adsabs.harvard.edu/abs/2020ApJ...899..158K}, \arXiv{https://arxiv.org/abs/2008.02518}} % ADS & arxiv links

%\publication
%	{The molecular ISM in the Super Star Clusters of the starburst NGC253} % Title
%	{\textbf{N. Krieger}, A. Bolatto, A. Leroy, R. Levy, E. Mills, D. Meier, S. Veilleux, F. Walter, A. Wei\ss} % Authors
%	{2020} % Year
%	{The Astrophysical Journal Vol.897, Issue 2, id.176} % Journal
%	{\ADS{https://ui.adsabs.harvard.edu/abs/2020ApJ...897..176K}, \arXiv{https://arxiv.org/abs/2006.08262}} % ADS & arxiv links

%\publication
%	{The Molecular Outflow in NGC\,253 at a Resolution of Two Parsecs} % Title
%	{\textbf{N. Krieger}, A. Bolatto, F. Walter, A. Leroy, L. Zschaechner, D. Meier, J. Ott, A. Wei\ss, E. Mills, S. Veilleux, M. Gorski} % Authors
%	{2019} % Year
%	{The Astrophysical Journal Vol.881, Issue 1, article id. 43, 20 pp} % Journal
%	{\ADS{https://ui.adsabs.harvard.edu/abs/2019ApJ...881...43K}, \arXiv{https://arxiv.org/abs/1907.00731}} % ADS & arxiv links

\vspace*{\fill}
Autorizzo il trattamento dei miei dati personali presenti nel curriculum vitae ai sensi del Decreto Legislativo 30 giugno 2003, n. 196 e del GDPR (Regolamento UE 2016/679).

\end{document}
