\documentclass[%singlesided,
               doublesided,
               paper=a4,
               fontsize=10pt
              ]{Rubini-Mattia-resume}


%%%%%%%%%%%%%%%%%%%%%%%%%%%%%%%%%%%%%%%%%%%%%%%%%%%%%%%%%%%%%%%%%%%%%%%%%%%%%%%%
% set geometry
%%%%%%%%%%%%%%%%%%%%%%%%%%%%%%%%%%%%%%%%%%%%%%%%%%%%%%%%%%%%%%%%%%%%%%%%%%%%%%%%

\setlength\highlightwidth{8cm}
\setlength\headerheight{4cm}            % note that margintop gets added to this value, i.e. the header bar is 5cm
\setlength\marginleft{1cm}
\setlength\marginright{\marginleft}      % needs to be 1.5 times to be actually equal. why?
\setlength\margintop{1cm}
\setlength\marginbottom{1cm}


%%%%%%%%%%%%%%%%%%%%%%%%%%%%%%%%%%%%%%%%%%%%%%%%%%%%%%%%%%%%%%%%%%%%%%%%%%%%%%%%
% FONTS
%%%%%%%%%%%%%%%%%%%%%%%%%%%%%%%%%%%%%%%%%%%%%%%%%%%%%%%%%%%%%%%%%%%%%%%%%%%%%%%%

\RequirePackage{fontspec}
\setmainfont{Carlito}


%%%%%%%%%%%%%%%%%%%%%%%%%%%%%%%%%%%%%%%%%%%%%%%%%%%%%%%%%%%%%%%%%%%%%%%%%%%%%%%%
% COLORS
%%%%%%%%%%%%%%%%%%%%%%%%%%%%%%%%%%%%%%%%%%%%%%%%%%%%%%%%%%%%%%%%%%%%%%%%%%%%%%%%

\colorlet{highlightbarcolor}{lightgray}
\colorlet{headerbarcolor}{darkgray}

\colorlet{headerfontcolor}{white}
\colorlet{accent}{awesome-red}
\colorlet{heading}{black}
\colorlet{emphasis}{black}
\colorlet{body}{black}


%%%%%%%%%%%%%%%%%%%%%%%%%%%%%%%%%%%%%%%%%%%%%%%%%%%%%%%%%%%%%%%%%%%%%%%%%%%%%%%%
% set document
%%%%%%%%%%%%%%%%%%%%%%%%%%%%%%%%%%%%%%%%%%%%%%%%%%%%%%%%%%%%%%%%%%%%%%%%%%%%%%%%


\begin{document}

\name{Mattia Rubini DevOPS}
\tagline{Il bello dei computer è che fanno esattamente quello che gli dici di fare. 
	\\Il brutto dei computer è che fanno esattamente quello che gli dici di fare.}
\photo[round]{photo-Mattia.jpg}{\dimexpr \headerheight-\marginbottom}   % make photo exactly match the header with margintop/marginright/marginbottom as margin

\makeheader

\highlightbar{

    \section{Contatti}
    
    \email{mattia.rubini@gmail.com}
    \phone{+39 329 617 1153}
    \location{Via Lionello Spada 30/A, 40129 Bologna}
    \vspace{0.5em}
    \github{Mot93}{https://github.com/Mot93}
    \linkedin{Mattia Rubini}{https://www.linkedin.com/in/mattia-rubini-1b8b49112/}
    
    \section{Skills}
    
    \skillsection{Programming}
    \skill{Python}{4}
    \skill{Rust}{4}
    \skill{SQL}{5}
    \skill{Bash}{4}
    \skill{C}{2}
    \skill{Java}{1}
    \skill{JavaScript}{3}
    \skill{HTML/CSS}{3}
    \skill{LaTeX}{2}
    
    \vspace{0.5em}
    \skillsection{Operating Systems}
    \skill{Linux}{4}
    \skill{MacOS}{4}
    \skill{Windows}{4}
    
    \vspace{0.5em}
    \skillsection{Software \& Tools}
    \skill{Git}{3}
    \skill{Jupyter}{4}
    \skill{Ansible}{4}
    \skill{Data handling/analysis}{4}
    (e.g. numpy,  pandas, ...)\\
    \skill{Docker}{5}
    \skill{Office}{3}
    
    \vspace{0.5em}
    \skillsection{Frameworks}
    \skill{Python Django}{4}
    \skill{Rust Rocket}{2}
    \skill{Dart Flutter}{2}

    \vspace{0.5em}
    \skillsection{Altre skills}
    \skill{Ubiquiti}{4}
    \skill{Raspberry Pi}{4}
    \skill{Arduino}{1}
    \skill{Assemblamento computer}{4}
    \skill{KNX}{3}
    \skill{BMS}{3}    

    \vspace{0.5em}
    \skillsection{Languages}
    \skill{Italian}{5}
    \skill{English}{4}
    \skill{French}{4}
    \bigskip
    
    %\section{Certificates}
    %\simpleskill{AWS certified cloud practitioner}

}
\mainbar{
    \section{Presentazione}
    	Sono un laureando in Informatica all'università di Bologna, e mi dedico a progetti di programmazione dall'età di 14 anni.\\
	In tutto quello che faccio, cerco sempre di migliorarmi e di imparare cose nuove.
	Tutti i progetti realizzati fino ad oggi sono frutto sia del mio lavoro da autodidatta che degli approfondimenti acquisiti nel corso dei miei studi di laurea.
	\\Questo curriculum è stato realizzato con \LaTeX, un linguaggio markup:
	il sorgente si trova presso il mio \href{https://github.com/Mot93/CV-Mattia-Rubini}{account gitHub}\\
    
    \section[\faGears]{Esperienze Lavorative}
    \job{Giu 2019 - Oggi}
        {MAIS Family Office}
        {System Integrator}
        {Impiegato presso il MAIS family office ho incominciato il percorso di System Integrator. 
	Finita la mia formazione, ho incominciato ad espandermi in altri ruoli. 
	Mi sono appassionato della figura del Network Engineer, System Administrator e Data Science.
	Aiutando i miei colleghi impiegati in questi settori, ho acquisito nuove conosceze e sono cresciuto anche nei loro ambiti.}
    
    \section[\faMortarBoard]{Formazione}
    \job{20020 - Oggi}
        {Alma Mater Studiorum, Bologna}
        {Laureando in Informatica}
        {Corso di studi in scienze informatiche.}
    \job{20017 - 2020}
        {Alma Mater Studiorum, Bologna}
        {Laureando in Ing. Informatica}
        {Interrotto, per trasferimento ad Informatica. superati 10 esami.}
    \job{20014 - 2015}
        {Alma Mater Studiorum, Bologna}
        {Borsa di studio a Shangai}
        {Gemelaggio tra Ing. dell'Automazione di Bologna e Shangai}
    \job{20013 - 2016}
        {Alma Mater Studiorum, Bologna}
        {Laureando in Ing. dell'automazione}
        {Interrotto, per trasferimento ad Ingegneria Informatica. superati 15 esami.}
    \job{2007 - 2013}
        {Liceo Niccolò Copernico, Bologna}
        {Diploma Liceo Scientifico}
        {Indirizzo di studi MAXI (potenziamento di liceo scientifico, su materie: matematica, fisica, programmazione e scienze).
	Tale indirizzo insegna a programmare fin dai primi anni.}

    %\section{General Skills}
    %\smallskip % additional skip because tag outlines use up space
    %\tag{Tag 1}
    %\tag{Tag 2}
    %\tag{and}
    %\tag{another tag}
    %\tag{some more tags}
    %\tag{yet another one}
    %\tag{tags flow over}
    %\tag{to the next line}
    %\tag{if necessary}
    
    %\medskip
    %Tags must be ordered by hand with newlines to get a nice layout, especially for long tags.
    
    %\section{Wheel Chart}
    % This is taken from AltaCV
    % see https://github.com/liantze/AltaCV for details
    %\wheelchart{1.5cm}{0.5cm}{% outer and inner diameter
    %    6/8em/accent!20/Sleep,          % comma-separated list of
    %    8/8em/accent!40/Daytime job,    % fraction of 24 / line length / color / label
    %    2/8em/accent!80/Training,          % here, the color is shades of the accent color
    %    3/8em/accent!60/Recovering from fighting criminals,
    %    5/8em/accent/Being Batman
    %}
}
\makebody
\clearpage


%\pagestyle{highlightmain}
%\highlightbar{}
%\mainbar{

    %\section{Another section}
    
    %This page uses the page style \texttt{highlightmain} which shows the highlight bar (gray) and the main part (white background) but omits the header. 
    %The default page style is \texttt{headerhighlightmain} with all three elements.
    %If you don't want header, nor highlight bar, use page style \texttt{\textbackslash pagestyle\{empty\}}.
    %\medskip
    %Neither main, nor highlight bar must be filled to make this template work.
    %It is possible to use a page style with the highlight bar but leave it empty by setting an empty highlightbar \texttt{\textbackslash highlightbar\{\}}.

    %\vspace{0.5em}
    %\subsection{Subsection 1}
    %Demonstrate subsections.
    
    %\subsection{Subsection 2}
    %Subsection are also bold face but a smaller font then section. They also omit the rule.
    

%}
%\makebody


%\clearpage
\pagestyle{empty}

%\section{Achievements, honours and awards}
    \section{Achievements}
    \achievement{Feb 2020 - Oggi\\
        Ho progettato e implementato la mia infrastruttura informatica e domotica usando prodotti \href{https://www.ui.com/}{Ubiquiti}, \href{https://www.qnap.com}{QNap} (NAS), \href{https://www.raspberrypi.org/}{Raspberry Pi} (home server e cluster) e \href{https://www.ikea.com/it/it/p/tradfri-gateway-bianco-40337806/}{Ikea}.
        La parte di networking, VPN, firewall, DNS e webserver sono pronte, sono ancora in cantiere la parte di automazione (\href{https://www.jenkins.io/}{Jenkins}), e la parte di microservizi (\href{https://kubernetes.io/}{Kubernetes}).
        \\La parte di domonica è in continua espansione
    }
    \achievement{Feb 2017 - Mag 2017, Postfix Calculator (app Android)\\
        Ho imparato a programmare app usando android studio, Java e XML. 
        A seguire ho pubblicato una app android \href{https://play.google.com/store/apps/details?id=postfixcalculator.mattiarubini.com.postfixcalculator}{ Postifix Calculator}.
        Il codice sorgente è pubblico su \href{https://gitlab.com/mattia.rubini/PostfixCalculator}{gitlab}
    }
    \achievement{Nov 2016 - Feb 2017, Wordpress (sito personale)\\
        Ho imparato a fare siti con wordpress. 
       Con queste nozioni ho fatto il mio sito personale. Il codice sorgente è disponibile su \href{https://github.com/Mot93/MattiaRubini-com-wordpress-theme}{github}.
    }

    \section{Motivazione}
        Non mi spavento facilmente davanti le avversità.
        Per non fermarmi davanti il primo ostacolo, a 16 anni ho imparato l'inglese (livello avanzato) per accedere a più materiale possibile.
        \\Questa mia indole, mista alla mia passione per l'informatica, mi ha permesso conoscere ed approfondire numerosi strumenti 
        grazie ai quali ho realizzato i numerosi progetti esposti sui miei account.

    \section{Altre Esperienze}
        Membro del \href{https://en.wikipedia.org/wiki/Rotaract}{Rotaract} Club Bologna Valle del Savena.
        \\\textbf{System administrator del distretto \href{https://rotaract2072.com/}{Rotaract 2072}.}
        \\Assemblo computer per uso personale, amici e familiari.
        \\ \textbf{Il mio interesse per Raspberry Pi e Arduino ha trovato applicazione sia nella supervisione degli apparati tecnologici del MAIS family office, sia per la mia bitazione.}
        \\Vacanza studio-lavoro di un mese con attività di volontariato per la ripopolazione della vegetazione in Islanda.
        \\Vacanza studio-lavoro di un mese con attività di volontariato per la preparazione di una struttura dedicata a ragazzi disadattati in Inghilterra.
        \\Attività di volontariato presso circolo Ancescao di Corticella in qualità di barista.
        \\Lavoro estivo in qualità di magazziniere in struttura a temperatura controllata, per generi alimentari.

%\section{Publications}
%\pubforcefullwidth

%Demonstrate what an \texttt{\textbackslash pagestyle\{empty\}} page looks like.
%Also show off the macros for publications that uses small icons for authors, date, journal and links.

%Achieving a good looking spacing can be tricky. For empty pagestyles where the full width is available use \texttt{\textbackslash pubforcefullwidth} to force the publoication list %to take up all the available space.
%The (relative) lengths reserved for date, journal and links can be set with the parameters \texttt{\textbackslash pubdatelength}, \texttt{\textbackslash pubjournallength} and \texttt{\textbackslash publinklength} as in \texttt{\textbackslash setlength\{\textbackslash pubdatelength\}\{0.15 \textbackslash linewidth\}}.
%\bigskip

%\publication
%	{The turbulent gas structure in the centers of NGC~253 and the Milky Way} % Title
%	{\textbf{N. Krieger}, A. Bolatto, E. Koch, A. Leroy, E. Rosolowsky, F. Walter, A. Wei\ss, D. Eden, R. Levy, D. Meier, E. Mills, T. Moore, J. Ott, Y. Su, S. Veilleux} % Authors
%	{2020} % Year
%	{The Astrophysical Journal Vol. 899, Issue 2, id.158} % Journal
%	{\ADS{https://ui.adsabs.harvard.edu/abs/2020ApJ...899..158K}, \arXiv{https://arxiv.org/abs/2008.02518}} % ADS & arxiv links

%\publication
%	{The molecular ISM in the Super Star Clusters of the starburst NGC253} % Title
%	{\textbf{N. Krieger}, A. Bolatto, A. Leroy, R. Levy, E. Mills, D. Meier, S. Veilleux, F. Walter, A. Wei\ss} % Authors
%	{2020} % Year
%	{The Astrophysical Journal Vol.897, Issue 2, id.176} % Journal
%	{\ADS{https://ui.adsabs.harvard.edu/abs/2020ApJ...897..176K}, \arXiv{https://arxiv.org/abs/2006.08262}} % ADS & arxiv links

%\publication
%	{The Molecular Outflow in NGC\,253 at a Resolution of Two Parsecs} % Title
%	{\textbf{N. Krieger}, A. Bolatto, F. Walter, A. Leroy, L. Zschaechner, D. Meier, J. Ott, A. Wei\ss, E. Mills, S. Veilleux, M. Gorski} % Authors
%	{2019} % Year
%	{The Astrophysical Journal Vol.881, Issue 1, article id. 43, 20 pp} % Journal
%	{\ADS{https://ui.adsabs.harvard.edu/abs/2019ApJ...881...43K}, \arXiv{https://arxiv.org/abs/1907.00731}} % ADS & arxiv links

\vspace*{\fill}
\href{https://www.overleaf.com/latex/templates/my-resume/qxsxdtmknkfr}{my-resume di Nico Krieger}

\end{document}
